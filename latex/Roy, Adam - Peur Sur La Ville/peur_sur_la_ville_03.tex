\chapter{Demain, la fin du monde ?}
Le lendemain\ldots{}

Quand le soleil se lève, la Chose revient dans le sous-sol. Elle se couche dans la caisse et s'endort. Elle fait toujours le même
rêve : elle voit un gar\c{c}on, il lui ressemble, il est jeune et beau. Il ouvre le couvercle de la caisse. Et\ldots{}

\og Écoute Alex, je pense que tu as l'âge de comprendre\ldots{} \fg{}

Alex ne répond pas, il n'a pas envie d'entendre la suite. Quand son père dit : \og tu as l'âge de comprendre \fg{}, c'est
toujours une mauvaise nouvelle. Mais monsieur Allard ne termine pas sa phrase. Quelqu'un sonne à la porte.

\og Je vais voir, papa ! \fg{}

Derrière la porte, un homme aux cheveux noirs et au visage souriant. Alex se dit qu'il l'a déjà vu quelque part. Il ressemble
à\ldots{} Il ne sait plus à qui.

L'homme en imperméable\footnote{Imperméable (n. m.) : manteau qui protège de la pluie.} lui tend la main.

\og Bonjour, je suis Félix Milder, du ministére de la Sécurité\footnote{Sécurité (n. f.) : sûreté.}. Je veux parler à Stéphane
Allard. \fg{}

Alex le fait entrer. Son père arrive.

\og Milder ! Je suis content de vous voir ! C'est terrible ce qui se passe ici.

--- C'est si grave que \c{c}a ?

--- Oui. \fg{}

L'homme baisse la voix.

\og Nous pouvons parler devant votre fils ?

--- Oui, il a âge de comprendre ! \fg{}

Encore cette phrase ! se dit Alex. Son père reprend.

\og Nous avons eu trois morts, la nuit dernière. Et combien d'habitants vont mourir, la nuit prochaine ? Nous ne savons plus quoi
faire. Vous pouvez nous aider.

--- Je vais essayer. Mais je veux voir les victimes. Où sont-elles ?

--- À la mairie. C'est juste devant. Venez, je vais vous montrer. \fg{}

Stéphane Allard croise le regard de son fils. Il sent qu'Alex veut lui demander quelque chose.

\og \ldots{} Alex, tu peux venir avec nous.

--- D'accord ! J'arrive ! \fg{}

Alex met un pull. Il veut savoir. Après tout, à quinze ans, il a âge de comprendre, comme dit son père. Bizarrement il s'attend au
pire.

Et il a raison\ldots{}

Même en plein jour, la ville est vide. Trois ombres traversent la place de la mairie : deux hommes et un jeune gar\c{c}on aux
cheveux noirs.

\og Après le vieux clochard, explique monsieur Allard, il y a eu ving-sept victimes. Deux ou trois chaque nuit : des hommes, des
femmes, des jeunes, des vieux. La secrétaire de la mairie était la troisième victime. Elle voulait me dire quelque chose, au sujet
de monsieur Schuwert. Je ne saurai jamais quoi\ldots{}

--- Ce n'est donc pas un tueur en série, dit Milder. Ils choisissent toujours les mêmes victimes.

--- Vous avez raison, c'est autre chose\ldots{} Un animal peut-être ? \fg{}

Alex pense aux extraterrestres.

Dans l'entrée de la mairie, personne. Tout est silencieux. Alex veut suivre les deux hommes, mais son père lui montre une chaise,
près d'une porte.

\og Toi, mon gar\c{c}on, tu nous attends là. Tu as l'âge de comprendre, mais pas de tout voir. \fg{}

Alex grogne, il n'est pas content. Il pensait trouver une trace des extraterrestres quelque part, sur les victimes\ldots{}

Son père ouvre la grande salle.

\og Suivez moi, Milder, s'il vous plaît. \fg{}

On entend des voix à travers la porte. Celle de son père et celle de Milder. Alex écoute. Et ce qu'il entend le fait trembler.

\og Voici les trois morts de la nuit dernière\ldots{} annonce monsieur Allard.

--- Ils ont été vidés de leur sang ?

--- Jusqu'à la dernière goutte ! Je n'ai jamais vu \c{c}a\ldots{} Regardez leur gorge.

--- Mais, ils ont été mordus !

--- Et là, regardez bien, on voit des traces\ldots{} Des traces de dents. Et \c{c}a ne ressemble pas aux dents d'un animal\ldots{}

--- Ni d'un homme !

--- Alors qu'est-ce que c'est ? demande Stéphane Allard d'une voix inquiète.

--- Je n'ose pas vous dire ce que j'en pense\ldots{} Pas ici, en tout cas.

--- Vous avez raison, allons dans mon bureau. \fg{}

Les deux hommes s'enferment dans le bureau et Alez n'entend plus rien. Mais il en sait assez. Les morsures\footnote{Morsure
(n. f.) : blessure faite en mordant.}\ldots{} Ce ne sont pas des extraterrestres, mais des vampires ! Et c'est pire.

Il se lève de sa chaise, regarde autour de lui. Personne. Il va jusqu'au bout du couloir. Alex voit la porte du sous-sol restée
ouverte. Il peut descendre par un petit escalier, comme dans ses jeux vidéos préférés. Il n'hésite pas\ldots{}

La Chose sait qu'il viendra. Elle l'attend, au fond de son sommeil. Bien sûr, elle va le mordre. Une seule morsure suffit. Et tous
les deux, ils seront invincibles\footnote{Invincible (adj.) : qui ne peut pas être vaincu.}\ldots{}

\og Alors, qu'en pensez vous, Milder ? \fg{}

Debout devant la fenêtre, Félix Milder réfléchit. Assis à son bureau, monsieur Allard attend qu'il parle.

\og Cette chose\ldots{} n'est pas un animal, ni un être humain, annonce Milder.

--- Quoi, alors ?

--- C'est une goule.

--- Une goule ?

--- Une femme vampire.

--- Une vampire ? Vous croyez à ces bêtises-là ?

--- Ce ne sont pas des bêtises et c'est très grave ! La nuit, elle tue pour se nourrir, mais bientôt\ldots{} Elle voudra un autre
    vampire avec elle. Elle va mordre quelqu'un, quelqu'un qui lui ressemble, sans le vider de son sang. Et ce vampire mordra
    d'autres gens, qui en mordront d'autres\ldots{}

--- C'est donc la fin du monde ?

--- Peut-être\ldots{} Si cette goule arrive à sortir de la ville, c'est la fin des êtres humains !

Monsieur Allard est inquiet. Quel terrible danger ! Pour lui, pour son fils, pour les habitants de Niedelbruck et pout tous les
êtres humains. \fg{}

Milder reprend :

\og La goule se cache quelque part. Peut-être près d'ici. Le jour, elle dort. Il faut la retrouver avant la nuit et la tuer.

--- Eh bien, on va chercher dans toute la ville ! Je vais prévenir l'armée. \fg{}

Stéphane Allard prend son téléphone.

\og Et comment peut-on la tuer, votre goule ?

--- Dites aux militaires de mettre des balles en argent dans leurs armes. Il y a aussi la lumière : le soleil tue les vampires.
\fg{}

La Chose sourit dans son sommeil. Une seule morsure\ldots{} Il sera immortel\footnote{Immortel(le) (adj.) : qui ne peut pas
mourir.}, comme elle. Et tous les deux, ils deviendront les maîtres du monde\ldots{}
