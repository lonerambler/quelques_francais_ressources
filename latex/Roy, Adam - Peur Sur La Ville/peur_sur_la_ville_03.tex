\chapter{Demain, la fin du monde ?}
Le lendemain\ldots{}

Quand le soleil se lève, la Chose revient dans le sous-sol. Elle se couche dans la caisse et s'endort. Elle fait toujours le même
rêve : elle voit un gar\c{c}on, il lui ressemble, il est jeune et beau. Il ouvre le couvercle de la caisse. Et\ldots{}

\og Écoute Alex, je pense que tu as l'âge de comprendre\ldots{} \fg{}

Alex ne répond pas, il n'a pas envie d'entendre la suite. Quand son père dit : \og tu as l'âge de comprendre \fg{}, c'est
toujours une mauvaise nouvelle. Mais monsieur Allard ne termine pas sa phrase. Quelqu'un sonne à la porte.

\og Je vais voir, papa ! \fg{}

Derrière la porte, un homme aux cheveux noirs et au visage souriant. Alex se dit qu'il l'a déjà vu quelque part. Il ressemble
à\ldots{} Il ne sait plus à qui.

L'homme en imperméable\footnote{Imperméable (n. m.) : manteau qui protège de la pluie.} lui tend la main.

\og Bonjour, je suis Félix Milder, du ministére de la Sécurité\footnote{Sécurité (n. f.) : sûreté.}. Je veux parler à Stéphane
Allard. \fg{}

Alex le fait entrer. Son père arrive.

\og Milder ! Je suis content de vous voir ! C'est terrible ce qui se passe ici.

--- C'est si grave que \c{c}a ?

--- Oui. \fg{}

L'homme baisse la voix.

\og Nous pouvons parler devant votre fils ?

--- Oui, il a âge de comprendre ! \fg{}

Encore cette phrase ! se dit Alex. Son père reprend.

\og Nous avons eu trois morts, la nuit dernière. Et combien d'habitants vont mourir, la nuit prochaine ? Nous ne savons plus quoi
faire. Vous pouvez nous aider.

--- Je vais essayer. Mais je veux voir les victimes. Où sont-elles ?

--- À la mairie. C'est juste devant. Venez, je vais vous montrer. \fg{}

Stéphane Allard croise le regard de son fils. Il sent qu'Alex veut lui demander quelque chose.

\og \ldots{} Alex, tu peux venir avec nous.

--- D'accord ! J'arrive ! \fg{}

Alex met un pull. Il veut savoir. Après tout, à quinze ans, il a âge de comprendre, comme dit son père. Bizarrement il s'attend au
pire.

Et il a raison\ldots{}

Même en plein jour, la ville est vide. Trois ombres traversent la place de la mairie : deux hommes et un jeune gar\c{c}on aux
cheveux noirs.

\og Après le vieux clochard, explique monsieur Allard, il y a eu ving-sept victimes. Deux ou trois chaque nuit : des hommes, des
femmes, des jeunes, des vieux. La secrétaire de la mairie était la troisième victime. Elle voulait me dire quelque chose, au sujet
de monsieur Schuwert. Je ne saurai jamais quoi\ldots{}

--- Ce n'est donc pas un tueur en série, dit Milder. Ils choisissent toujours les mêmes victimes.

--- Vous avez raison, c'est autre chose\ldots{} Un animal peut-être ? \fg{}

Alex pense aux extraterrestres.

Dans l'entrée de la mairie, personne. Tout est silencieux. Alex veut suivre les deux hommes, mais son père lui montre une chaise,
près d'une porte.

\og Toi, mon gar\c{c}on, tu nous attends là. Tu as l'âge de comprendre, mais pas de tout voir. \fg{}
