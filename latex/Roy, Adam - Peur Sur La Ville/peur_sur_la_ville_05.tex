\chapter{L'Âme s\oe{}ur}
Le lendemain\ldots{}

Sur le mur, la photo du Président. Et une grande table. Félix Milder est assis au milieu. À sa droite, Stéphane Allard et à sa
gauche, son fils Alex. Devant eux, la salle de la mairie est pleine de monde.

Milder explique aux journalistes ce qu'est une goule et quels sont ses pouvoirs.

\og C'était un monstre, un monstre dangereux\ldots{}

--- Et ce monstre\footnote{Monstre (n. m.) : animal imaginaire qui fait peur.} a tué tous ces gens ? demande un vieux monsieur.

--- Oui, la goule avait besoin de sang pour se nourrir.

--- Qu'est-ce qu'elle voulait ?

--- La goule cherchait son âme s\oe{}ur\ldots{}

--- Et après ? demande une jeune femme.

--- Après, le règne\footnote{Règne (n. m.) : temps pendant lequel quelqu'un exerce son pouvoir.} des vampires commencer.

--- On a eu de la chance ! \fg{} murmure une voix.

Milder regarde Alex. Un gar\c{c}on courageux\footnote{Courageux(se) (adj.) : qui a la volonté de se battre face à un danger.}. Il
a eu beaucoup de chance !

Stéphane Allard prend la parole. Il explique que la goule est bien morte. L'armée a tout nettoyé. Il n'y a plus de vampire, à
Niedelbruck.

\og Et si \c{c}a recommence, dans une autre ville ? demande quelqu'un.

--- Impossible. On ne peut plus vendre ni acheter de la terre des Carpates. Elle est interdite. Il n'y a plus à avoir peur.

--- Tant mieux ! \fg{} soupire quelqu'un.

Alex a un peu peur. Les micros se tendent vers lui, comme des serpents qui dressent la tête. Il y a aussi les flashes des
photographes.

\og Et vous, Alex, comment avez-vous trouvé le monstre ? demande une journaliste de la télé.

--- C'est á cause de \emph{Legacy of Kane}, le jeu vidéo.

--- Mais comment\ldots{}

--- La première fois que je suis allé au sous-sol, je n'ai rien vu, explique Alex. Il y avait des vieilles choses et une caisse
    de terre, dans un coin. Quand je suis rentré chez moi, j'ai allumé ma console de jeu.

--- Et alors ?

--- Au troisième niveau du jeu, Kane trouve la goule, son ennemie de toujours, dans un cimetière des Carpates. J'ai tout de suite
    pensé à la caisse de terre. Alors, j'ai deviné qu'elle était là-bas, bien cachée. \fg{}

Alex n'avoera\footnote{Avouer (v.) : reconnaître que quelque chose est vrai.} pas qu'il n'a rien deviné, non, il l'a sentie. Elle
l'appelait, elle l'attirait\ldots{} Il devait revenir au sous-sol.

\og À quoi ressemblait-elle ?

--- Je n'ai vu que ses yeux, des yeux terribles. Et sa bouche, si rouge. Des dents longues, longues\ldots{} \fg{}

Il y a des choses qu'Alex ne dira jamais à personne. Et surtout pas aux journalistes. Il ne leur dira pas que la goule ressemblait
à sa s\oe{}ur. Les mêmes cheveux noirs, le même visage\ldots{} Il a cru que Marine avait trouvé un moyen d'entrer dans la ville

\og Marine, mais qu'est-ce que tu fais là ? \fg{}

Elle l'a regardé. Ses yeux de feu brllaient. Une lumière surnaturelle\footnote{Surnaturel(le) (adj.) : qui n'appartient pas à la
réalité.}. Sa bouche, rouge, si rouge, s'est ouverte. Alors, Alex a vu ses dents, si longues. Ce n'était pas Marine, non, mais
autre chose\ldots{} Il a eu peur, très peur. Mais il ne pouvait plus bouger. Paralysé\footnote{Paralysé (adj.) : qui ne peut pas
bouger.}.

Les micros sont toujours tendus vers lui.

\og Alex, racontez-nous ce qui s'est passé, au sous-sol\ldots{} demande quelqu'un.

--- Oui, racontez-nous tout ! dit un autre.

--- Je veux une interview pour mon magazine !

--- Non, c'est pour moi, je suis arrivé le premier ! \fg{}

\ldots{} missing more text here \ldots{}

\og Non ! Je ne veux pas, je ne veux pas ! \fg{}

Et il a enfin réussi à bouger.

Alors, il a tiré et poussé la lourde caisse, devant une petite fenêtre. Vers le dernière rayon de lumière. Il pleurait, quand elle
s'est enflammée\footnote{Enflammer (v.) : brûler.}.
