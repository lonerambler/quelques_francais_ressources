\chapter{La Chose}
\textbf{Mais, ces temps-ci, la Chose dort mal. Des bruits la dérangent. La terre bouge autour d'elle. Le monde bouge. Au fond de
son sommeil, elle sent que l'Heure arrive\ldots{}}

Niedelbruck est une ville tranquille, au Nord de la France. On y trouve des maisons fleuries, des jardins fleuris,
des rues fleuries et, sur la place de la mairie, des fleurs de toutes les couleurs. Il ne se passe jamais rien ici.
Et, ce mercredi de septembre, tous les habitants regardent un camion noir\ldots{} Il s'arrête devant la mairie. Le
chauffeur descend, un papier à la main, et entre dans le bâtiment.

\textbf{Oui, l'Heure arrive. La Chose le sait, elle le sent. Et elle se prépare à sortir. Alors, le monde sera à elle\ldots{}}

La secrétaire secoue la tête.

\og{} Non, nous n'avons pas de monsieur Schuwert, ici. \fg{}

Le chauffeur du camion s'étonne :

\og{} Mais j'ai une caisse pour un certain René Schuwert, fleuriste à Niedelbruck. C'est bien ici ? \fg{}

La femme réfléchit un moment.

\og{} Ah oui ! Je vois de qui vous voulez parler. Un vieil home bizarre\footnote{Bizarre (adj.) : étrange, curieux.}
qui adorait les orchidées\footnote{Orchidée (n. f.) : plante aux fleurs à trois pétales.}. Il est mort, l'année
dernière.

--- Alors, qu'est-ce je vais faire, moi, avec ma caisse ? Il faut que je la laisse quelque part ! \fg{}

La femme demande :

\og Et qu'est qu'il y a dans cette caisse ?

--- De la terre des Carpates\footnote{Carpates (n. f. pl.) : chaîne de montagnes à l'Est de l'Europe.}.

--- Ah ?

--- Oui, del a terre de cimetière\footnote{Cimetière (n. m.) : lieu où les morts son enterrés.}. C'est très bon pour
    les orchidées. C'est aussi très, très cher ! Il y en a pour des milliers d'euros\ldots{}

--- Eh bien, mettez vous votre caisse ici, dans le sous-sol\footnote{Sous-sol (n. m.) : situé au-dessous du sol, cave.}.
    Je vais prévenir monsieur le maire. Il téléphonera au fils de monsieur Schuwert. Il viendra la chercher.

--- D'accord, j'y vais ! \fg{}

Derrière leurs fenêtres, les habitants de Niedelbruck regardent le chauffeur descendre la caisse du camion. Une grande
caisse de bois, très lourde. Le camion repart et plus rien ne se passe. La vie continue comme tous les jours, à
Niedelbruck.

\textbf{La Chose se réveille enfin. Elle ouvre les yeux. Autour d'elle, c'est la nuit. Elle a dû dormir très longtemps. Et
maintenant, ella a faim. Très, très faim. Elle soulève le couvercle et sort de sa caisse.}

Il est tard, Niedelbruck s'endort. C'est une belle nuit d'automne, toutes les fenêtres sont ouvertes.

Dans sa chambre, Alex a allumé sa console de jeux\footnote{Console de jeux (n. f.) : ordinateur pour jouer aux jeux
vidéo.}. Dans le jeu vidéo, il est Kane, le chasseur de vampires\footnote{Vampire (n. m.) : créature qui se nourrit du
sang des vivants.}. Il avance dans des couloirs sombres, son épée à la main. Quand un vampire l'attaque, il lui enfonce
son épée dans le c\oe{}ur. Alex-Kane s'amuse comme un fou !

Son père passe sa tête par la porte.

\og Alex, il est tard !

--- Encore cinq minutes, papa, s'il te plaît ! Après, je vais dormir, je te le promets.

--- D'accord\ldots{} Mais pas plus de cinq minutes !

--- Promis, papa. Croix de bois, croix de fer, si je mens, je vais en enfer ! \fg{}

Stéphane Allar sourit. Il referme la porte de la chambre.

\textbf{La Chose n'a pas mangé depuis de longues années. Elle sent une odeur de sang, tout près d'elle. Couché devant une porte,
un clochard\footnote{Clochard (n. m.) : personne sans travail qui vit dans la rue.} dort. La Chose s'approche de lui,
comme une ombre\ldots{}}

Alex tue son dernier adversaire. Il éteint la console et se tourne vers la fenêtre ouverte. Il sursaute. Il voit une
ombre sortir doucement du sous-sol. L'ombre traverse la place de la mairie. Et puis, il entend un cri bizarre.

Alex secoue la tête. Il murmure :

\og Je ne jouerai plus aussi tard ! Je vois des vampires partout, maintenant ! \fg{}

\textbf{La Chose boit le sang du clochard jusqu'à la dernière goutte. Mais elle a encore faim. Alors, elle cherche une autre
victime\footnote{Victime (n. f.) : personne tuée.} dans les rues sombres\footnote{Sombre (adj.) : avec peu de lumière.}
de la ville\ldots{}}
