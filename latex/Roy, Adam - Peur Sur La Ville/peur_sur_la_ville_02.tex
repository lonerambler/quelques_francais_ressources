\chapter{Un tueur dans la ville}
Quinze jours plus tard\ldots{}

Alex ouvre la porte.

\og Papa ? Papa, je suis là ! \fg{}

Mais personne ne répond. La maison est vide. Son père n'est pas encore rentré. C'est normal, pense Alex, avec tout ce qui se
passe. Monsieur Allard est le maire de la ville. Alors, avec tous ces meurtres\footnote{Meurtre (n. m.) : tuer volontairement une
personne.} bizarres, on a besoin de lui. Il y a des militaires\footnote{Militaire (n. m.) : personne de l'armée.} partout, ils ne
laissent plus entrer personne à Niedelbruck, et personne ne peut plus sortir de la ville.

Alex va dans sa chambre, pose son sac à côté du bureau. Sur le mur, une photo, deux jeunes qui se ressemblent. Ils se tiennent par
le bras. Ils sourient. Alex pense à sa s\oe{}ur, elle lui manque. Elle est partie en voyage, aver le lycée, et elle est restée de
l'autre côté.

Alex descend, il prépare son goûter et allume la télé. Derrière son micro, le journaliste parle d'un ton grave :

\og Peur sur la petite ville de Niedelbruck. La nuit dernière, le tueur en série\footnote{Tueur en série (n. m.) : criminel qui tue
de fa\c{c}on répétitive.} a fait trois nouvelles victimes. Les habitants de Niedelbruck ne doivent pas sortir après dix-huit
heures. Je répète, restez chez vous dès que la nuit tombe ! Vous risquez votre vie\footnote{Risquer sa vie : mettre sa vie en
danger.} ! \fg{}

Alex a peur. Persone ne sait vraiment ce qui arrive. Il y a des militaires plein les rues. Mais les meurtres continuent. Toutes
les nuits, de nouvelles victimes. Maintenant, le soir arrive et son père n'est toujours pas là\ldots{}

La Chose se réveille. La nuit tombe. Elle a faim. Une faim terrible\ldots{} Depuis quelques jours, elle ne trouve plus de
nourriture, les habitants se cachent. Il y a aussi des hommes, des armes, des chiens. Elle a faim, elle a vraiment faim.

Au lycée, on ne parle plus que du tueur. Peut-être un fou, sorti d'une prison ou d'un hôpital ?

Alex, lui, a une autre idée. Il pense que ce n'est pas un homme, mais un extraterrestre. Il a vu dans un film, que les
extraterrestres veulent envahir la Terre. Ils ont besoin de sang humain pour leurs expériences\footnote{Expérience (n. f.) :
essais pour étudier, comprendre quelque chose.}\ldots{}

Dix-neuf heures sonnent et son père n'arrive pas ! Alex est inquiet\footnote{Inquiet(ète) (adj.) : qui a peur que quelque chose se
passe mal.}. Fou ou extraterrestre, son père ne sera pas la prochaine victime !

Une sonnerie retentit, deux sonneries, trois sonneries. Alex se précipite vers le téléphone.

\og Allô ? C'est toi, Marine ?\ldots{} Oui, tout va bien\ldots{} Non, papa n'est pas là\ldots{} Il téléphonera\ldots{} Oui, moi
aussi, t'en fais pas ! \fg{}

Quelqu'un, enfin ! Un homme traverse la place. La Chose le suit cachée dans l'ombre. Ell va lui sauter à la gorge\footnote{Gorge
(n. f.) : partie avant du cou.}\ldots{} Mais soudain, l'homme passe dans la lumière. Elle recule. Puis il ouvre une porte et
disparaît. La Chose grogne. Elle s'enfuit dans la nuit.

\og Alex, tu es là ? \fg{}

Le gar\c{c}on sursaute. Son père, enfin ! Il se précipite dans l'entrée.

\og Bonsoir papa ! \fg{}

Monsieur Allard enlève sa veste. Il semble fatigué.

\og Tu as l'air inquiet, papa.

--- Un peu. Ta s\oe{}ur est hors de danger\footnote{Danger (n. m.) : un risque grave.} mais pas toi. Si je pouvais te faire sortir
    de la ville\ldots{}

--- Tu sais ce qui se passe, ici ?

--- J'ai mon idée.

--- Et c'est grave ?

--- Oui. \fg{}

Alex regarde le visage de son père, il comprend que ce n'est pas la peine d'insister.

La Chose passe devant une vitrine. Elle se regarde, mais ne voit rien, elle n'a pas de reflet\footnote{Reflet (n. m) : image de
soi renvoyée par une surface (eau, miroir).}. Pourtant, elle sait qu'elle est jolie : elle ressemble à une fille de quinze ans,
aux cheveux noirs et aux yeux de feu. Ses longues dents brillent dans la lumière. Soudain, elle dresse la tête. Elle a entendu
quelque chose\ldots{}
