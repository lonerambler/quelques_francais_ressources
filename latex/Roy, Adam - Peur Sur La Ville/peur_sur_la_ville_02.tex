\chapter{Un tueur dans la ville}
Quinze jours plus tard\ldots{}

Alex ouvre la porte.

\og Papa ? Papa, je suis là ! \fg{}

Mais personne ne répond. La maison est vide. Son père n'est pas encore rentré. C'est normal, pense Alex, avec tout ce qui se
passe. Monsieur Allard est le maire de la ville. Alors, avec tous ces meurtres bizarres, on a besoin de lui. Il y a des militaires
partout, ils ne laissent plus entrer personne à Niedelbruck, et personne ne peut plus sortir de la ville.

Alex va dans sa chambre, pose son sac à côté du bureau. Sur le mur, une photo, deux jeunes qui se ressemblent. Ils se tiennent par
la bras. Ils sourient. Alex pense à sa s\oe{}ur, elle lui manque. Elle est partie en voyage, aver le lycée, et elle est restée de
l'autre côté.

Alex descend, il prépare son goûter et allume la télé. Derrière son micro, le journaliste parle d'un ton grave :

\og Peur sur la petite ville de Niedelbruck. La nuit dernière le tueur en série a fait trois nouvelles victimes. Les habitants de
Niedelbruck ne doivent pas sortir après dix-huit heures. Je répète, restez chez vous dès que la nuit tombe ! Vous risquez votre
vie !\fg{}

Alex a peur. Persone ne sait vraiment ce qui arrive. Il y a des militaires plein les rues. Mais les meurtres continuent. Toutes
les nuits, de nouvelles victimes. Maintenant, le soir arrive et son père n'est toujours pas là\ldots{}

La Chose se réveille. La nuit tombe. Elle a faim. Une faim terrible\ldots{} Depuis quelques jours, elle ne trouve plus de
nourriture, les habitants se cachent. Il y a aussi des hommes, des armes, des chiens. Elle a faim, elle a vraiment faim.

Au lycée, on ne parle plus que du tueur. Peut-être un fou, sorti d'une prison ou d'un hôpital ?
