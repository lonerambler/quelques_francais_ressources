\chapter{Au sous-sol}
\og Alex, où est tu ? Alex ? Alex ? \fg{}

Stéphane Allard cherche son fils. Il n'est plus sur sa chaise, dans le couloir, ni devant la mairie\ldots{}

\og Mais, où est donc ce gar\c{c}on ? \fg{}

Milder reste calme\footnote{Calme (adj.) : sans agitation, tranquille.}.

\og Ne vous inquiétez pas, monsieur Allard, votre fils n'est pas loin. Et puis, il ne risque rien quand le soleil brille. C'est
la nuit que c'est dangereux.

--- Qu'allons-nous faire ?

--- Nous allons chercher et tuer ce vampire ! \fg{}

\textbf{Tout à coup, la Chose se réveille. Elle sent quelque chose. Il n'est pas loin. Il arrive. Il approche. Tout près, oui,
tout près d'elle !}

Le sous-sol est sombre. Très grand. Plein de poussière. Alex regarde autour de lui : des caisses de livres, des vieux
meubles\ldots{} Et une grande caisse dans un coin.

\og C'est bizarre, ici ! murmure-t-il. \fg{}

Le jeune gar\c{c}on fait quelques pas. Il hésite. Quels terribles ennemis vont sortir des coins sombres ? Des vampires assoiffés
de sang comme dans \emph{Legacy of Kain II} ? Soudain, Alex n'est plus Alex, mais Kane, le chasseur de vampires\ldots{}

\textbf{La Chose le sent. Il est à quelques mètres, il se rapproche encore\ldots{} Elle est bien là, son âme s\oe{}ur\footnote{Âme
s\oe{}ur (n. f.) : personne très proche.}. Elle l'attend depuis si longtemps\ldots{} Une simple morsure\ldots{} pour la rendre
éternelle\footnote{Éternel(le) (adj.) : indestructible, immortel.}}

L'ennemi n'est pas loin, Kane le sait, Kane le sent. Il sort sa longue épée, la \emph{Soul River}. Il fait trois pas dans le
sous-sol et s'arrête. Dans un coin sombre, une grande caisse en bois. Il s'approche.

\og Qu'est-ce que c'est que \c{c}a ? \fg{}

Alex regarde de plus près et lit les lettres presque effacées.

\og Comte Drac\ldots{} château de\ldots{} Moldavie. \fg{}

Une publicité annonce : \emph{Avec la terre de cimetière, vos orchidées sont fières !}

\og Une caisse de terre ? murmure Alex. Qu'est-ce qu'elle fait là ? \fg{}

Cette caisse l'attire\footnote{Attirer (v.) : Faire venir à soi.}. C'est bizarre\ldots{} Il avance, il s'approche. Il veut
soulever le couvercle\ldots{}

\textbf{Le couvercle bouge et la Chose attend, les yeux grands ouverts. Elle attend son âme s\oe{}ur. Une simple morsure, pour la
rendre immortelle\ldots{}}

La voix de son père retentit, au-dessus de sa tête.

\og Alex ! Alex ! \fg{}

Alex lâche le couvercle. Il n'est pas dans un jeu vidéo, non. Il est dans la réalité. Une terrible réalité.

\og J'arrive, papa, j'arrive ! \fg{}

Alex secoue la poussière sur son pull. Il remonte l'escalier.

\textbf{Les longues dents de la Chose mordent dans le vide. Son âme s\oe{}ur lui a échappé\ldots{} Mais elle reviendra bientôt,
elle le sait\ldots{} Elle l'appelle.}
