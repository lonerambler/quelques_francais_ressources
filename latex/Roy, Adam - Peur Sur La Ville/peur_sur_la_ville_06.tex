\chapter{Quinze jours plus tard\ldots{}}
L'armée est partie et les journalistes aussi. Niedelbruck a retrouvé son calme.

Enfin, presque. Dans un vieille maison, vers la sortie de la ville, un chauffeur charge\footnote{Charger (v.) : mettre quelque
chose de lourd dans un véhicule.} un camion.

Le fils de René Schuwert vide la maison de son père. Les meubles, les vieux souvenirs\footnote{Souvenir (n. m.) : ce qui permet
de se rappeler quelque chose.}, les orchidées disparaissent dans le camion noir. Maintenant, on vide le sous-sol.

\og Monsieur Schuwert ! Venez voir ! Qu'est-ce qu'on fait de \c{c}a ? \fg{}

Au fond, dans le coin le plus sombre, trois caisses oubliées attendent.

\og Qu'est-ce que c'est ?

--- De la terre des Carpates, je crois.

--- C'est interdit, maintenant ! dit un homme.

--- Oui, mais c'est cher, très cher, répond le fils. Allez, on les prend ! \fg{}

Les hommes sortent les caisses, ils les chargent sur le camion et grognent. Elles sont lourdes ! Trés, trés lourdes !

\c{C}a y est, c'est fini, le camion peut partir. Pour Arles, une petite ville tranquille du Sud de la France.
