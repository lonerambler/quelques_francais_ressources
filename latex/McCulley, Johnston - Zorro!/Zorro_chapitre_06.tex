\chapter{Frère Felipe}
Le capitaine Ramón revient au fort. Il est furieux.

--- Je dois punir Lolita, sa famille et Zorro ! Je veux écrire une lettre au gouverneur. Don Carlos Pulido est l'ami de Zorro. Il
    aide et protège ce bandit. C'est un traître\footnote{Un traître : personne non loyale.}.

Le capitaine fait donc une lettre au gouverneur. Il relit sa lettre et dit, à haute voix, avec une sourire 
mauvais\footnote{Mauvais : ici, méchant.} :

--- Je veux voir la famille de Don Carlos en prison.

Une voix derrière le capitaine dit :

--- Je veux vous voir, vous, en prison.

Surpris, Ramón se retourne et voit Zorro. Zorro dit encore :

--- Vous êtes un homme indigne, battez-vous avec moi, mais laissez tranquille la famille de Don Carlos !

Le capitaine va à la port et crie :

--- Sergent ! Sergent ! À l'aide ! Zorro est ici !

Mais quand il se retourne, la pièce est vide !

--- Me voici, capitaine ! dit le gros sergent.

--- Prenez vos hommes ! Il ne doit pas être loin. Nous devons capturer Zorro !

Les soldats partent sur les traces\footnote{Une trace : ici, marque, indice.} de l'homme masqué, mais il fait nuit et il est
difficile de suivre le cheval rapide de Zorro.

Le lendemain matin, les hommes de Gonzales rentrent au fort, sans Zorro. Ils sont fatigués et furieux. Zorro est encore en
liberté !

Il y a beaucoup de monde devant le fort quand ils rentrent. Don Diego est là, lui aussi. Il demande :

--- Pourquoi est-ce qu'il y a tant de monde ce matin ?

Un vieux moine est debout devant le juge. Il est enchaîné.

--- Je ne suis pas un voleur, déclare le frère, je suis seulement un pauvre moine !

--- Pourquoi frère Felipe est ici ? demande Don Diego au juge.

--- Le vieux moine est un voleur, il doit être puni ! répond le juge.

--- C'est impossible, déclare Don Diego, frère Felipe est un homme honnête, je connais frère Felipe depuis longtemps.

--- Non, vous vous trompez\footnote{Se tromper : être dans l'erreur.} ! assure le juge.

Et il appelle deux soldats.

--- Fouttez le moine quinze fois !

Les soldats donnent quinze coups de fouet. Le vieux moine tombe. Don Diego est en colère car frère Felipe est son ami. Il rentre à
l'hacienda.

--- Bonjour, mon fils, dit Don Alejandro Vega. Je suis heureux de pouvoir parler avec toi ! Qu'est-ce que tu penses de la fille de
    notre ami Don Carlos ? Est-ce qu'elle accepte de devenir ta femme ?

--- Lolita ? J'aime la fille de Don Carlos, mais elle aime seulement les hommes romantiques. Que puis-je faire ? dit Don Diego.

--- Les jeunes filles aiment les hommes courageux et romantiques ! Il faut parler d'amour avec Lolita. Il faut jouer de la guitare
sous son balcon. Les femmes aiment les fleurs et les chansons d'amour. Les jeunes gens romantiques font comme ça ! explique Don
Alejandro à son fils.

--- Mais c'est ridicule ! Je ne veux pas faire ça\ldots{}

--- Il faut essayer ! Lolita est une jeune fille charmante\ldots{} ajoute le père de Don Diego.

--- C'est trop compliqué ! Je vais me reposer !
