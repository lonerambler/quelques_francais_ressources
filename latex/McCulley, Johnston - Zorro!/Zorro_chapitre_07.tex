\chapter{Les justiciers}
Dans la soirée, le juge et ses amis sont à la taverne. Ils parlent du vieux moine et ils rient beaucoup.

--- Pourquoi est-ce que vous riez ? demande une voix.

Le juge et ses amis regardent vers la porte. Zorro est là, il regarde le juge.

--- Le frère Felipe n'est pas un voleur, vous savez très bien cela monsieur le juge, déclare Zorro. Je suis ici pour vous punir !

--- Je suis un juge très puissant et n'aime pas les moines, parce que ce sont vos amis.

Zorro donne un fouet à un ami du juge et dit :

--- Et maintenant, donnez quinze coups de fouet au juge !

--- Mais je ne peux pas faire une chose semblable ! s'écrie l'homme.

--- Fouettez le juge ou je vous fouette, vous !

Alors, l'homme fouette le juge. Après la punition, le juge tombe.

--- Voilà la punition des gens malhonnêtes, dit Zorro.

Le lendemain, le pays entier parle de la punition du juge.

Un groupe de jeunes gens décide d'aider le capitaine Ramón et de capturer Zorro. Ils cherchent dans les collines, dans les vallées
et vers le soir, ils s'approchent de l'hacienda de Don Alejandro.

--- Que voulez-vous ? demande Don Alejandro, surpris.

--- Nous sommes à la recherche de Zorro, nous voulons la récompense, mais il est tard et nous avons faim. Pouvons-nous manger ?

--- Posez vos armes près de la porte et venez avec moi ! Ces gâteaux et ce vin sont pour vous !

Don Alejandro, Don Diego et les jeunes gens bavardent\footnote{Bavarder : parler.} ensemble mais à neuf heures, Don Diego se lève
pour aller dormir.

--- Il est seulement neuf heures, mon fils, pourquoi est-ce que tu ne restes pas encore avec nous ?

--- Je suis fatigué, mon père.

--- Fatigué? Tu es toujours fatigué ! Regarde ces jeunes gens, ils ne sont pas fatigués !

--- Vous avez raison, mon père. Bonne nuit à tout le monde !

Les jeunes gens mangent, boivent et chantent.

À minuit, un homme masqué apparaît à la porte.

--- Zorro ! dit un des jeunes gens. Le bandit !

--- Mon nom est Zorro, mais je ne suis pas un bandit ! Je lutte contre les hommes corrompus et j'aide les pauvres, les faibles et
    les moines ! Et vous, pourquoi est-ce que vous luttez ?

--- Nous désirons aider les pauvres, les habitants originaires de Californie et les moines, mous aussi ! crie un jeunne homme.

--- Nos idées sont semblables ! dit un autre.

--- Alors, luttez avec moir ! Combattons pour la même cause !

--- Mais que êtes-vous ? Où habitez-vous ? demande un des jeunes gens.

--- Je ne peux rien dire, c'est un secret.

Le jeune homme continue.

--- Oui, nous voulons lutter avouc vous, nous voulons la justice en Californie. Nous sommes les justiciers !

--- Oui, oui, nous sommes les justiciers ! s'écrient les jeunes gens.

--- Et alors, luttons ensemble ! s'excalme Zorro.

Il quitte la pièce et disparaît dans la nuit.
