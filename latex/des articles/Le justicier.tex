\documentclass[a5paper,notitlepage]{article}
\usepackage[francais]{babel}
\usepackage[utf8]{inputenc}
\usepackage[pass]{geometry}

\author{Chantal Delaplanche}
\title{Le justicier}

\begin{document}
\maketitle
Le personnage du héros justicier est une figure traditionnelle populaire. Depuis toujours, le public aime l'image du
hors-la-loi imprenable qui défie avec talent, habileté et humour la police, les autorités et le mal. Parfois, il porte
un masque. Le justicier est toujours entouré de mystère et il mène une double vie. L'album de famille des bandits
honnêtes est très varié, mais le justicier par excellence est Zorro.

\section{Zorro}
McCulley, le journaliste et romancier qui raconte en 1919 ses aventures avec \emph{The Curse of Capistrano}, ne sait pas que le
cinéma va s'emparer de ce personnage de nombreuses fois ! Des acteurs très célèbres jouent le rôle de Zorro : Douglas Fairbanks,
Tyrone Power, Alain Delon, Antonio Banderas\ldots{}

Dans le film de Spielberg, \emph{La Légende de Zorro}, Antonio Banderas incarne le successeur de Zorro.

\section{Robin des Bois}
Nous sommes en Angleterre, au Moyen Âge. Le baron de Huntingdon, plus connu sous le nom de Robin des Bois, incarne la résistance
saxonne contre les envahisseurs normands.

Ce bandit, très habile avec un arc et des flèches, est très apprécié du public, car il lutte contre le pouvoir et ses abus, se
moque de la loi et met son intelligence, sa générosité et sa force au service de la bonne cause.

\end{document}
