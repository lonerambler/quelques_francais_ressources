\documentclass[a5paper,notitlepage]{article}
\usepackage[francais]{babel}
\usepackage[utf8]{inputenc}
\usepackage[pass]{geometry}

\author{Claude Louvet}
\title{Qui est Guy de Maupassant ?}

\begin{document}
\maketitle

Guy de Maupassant naît le 5 août 1850 au château de Miromesnil près de Dieppe, en Normandie. Sa famille est de
noblesse récente. Ses parents s'installent à Paris où Guy est lycéen mais à leur séparation, il revient avec sa mère
et son frère Hervé, en Normandie, à Étretat. Guy écrit déjà des vers, il est pensionnaire à Rouen. Plus tard il connaît
Gustave Flaubert, normand aussi. Après le baccalauréat\footnote{baccalauréat : examen qui termine les études
secondaires.} en 1869, il s'incrit à la faculté de droit à Paris, mais en 1870 il est appelé à la guerre contre les
Prussiens. Il va garder toute sa vie la vision des atrocités de la guerre.

Il quite l'armée et travaille au ministère de la Marine.

En 1873, commence pour Maupassant une grande période de dépression et de tristesse. Il écrit des contes sous la direction de
Flaubert. Le grand écrivain va avoir beaucoup d'influence sur lui et il va donner des règles précises sur l'art d'écrire.
Maupassant travaille beaucoup à ses histoires mais le dimanche il va canoter sur la Seine. On retrouve dans ses récits beaucoup
de scènes, d'impressions de ces dimanches en canot.

Chez Flaubert, il rencontre l'historien et écrivain Edmond de Goncourt et l'auteur dramatique russe Tourgueniev.

En 1875, il publie son premier conte : \emph{La Main d'Échorché}, sous un pseudonyme. Il écrit aussi des pièces de théâtre. Il
rencontre à cette époque, le poète Mallarmé et les écrivains Huysmans, Zola et Daudet. Il collabore au quotidien \emph{La nation}.

La santé de Maupassant se dégrade. En 1879, il quitte le ministère. Il participe au recueil \emph{Les Soirées de Médan} et publie
la nouvelle \emph{Boule de suif} qui lui apporte le succès et la richesse.

Maupassant publie surtout dans les journaux, dans \emph{Gil Blas}, \emph{La revue des deux mondes}, \emph{La vie populaire}\ldots{}

Maupassant a écrit environ trois cents contes réunis en une quinzaine de recueils comme \emph{La Maison Tellier}, les \emph{Contes
de la bécasse}, \emph{Miss Harriet} et \emph{La petite Roque}.

Sa santé empire. Il souffre beaucoup des yeux et a des troubles nerveux. Pendant ces années, Maupassant visite l'Afrique du Nord,
la Sicile et l'Angleterre et fait de longues traversées sur son yacht \emph{Bel Ami}. Il publie des récits de voyage, comme
\emph{La vie errante}.

À part ses recueils de contes, Maupassant, auteur, a écrit des romans importants : \emph{Une vie}, \emph{Bel Ami},
\emph{Mont-Oriol}, \emph{Notre cœur}. C'est aussi un grand journaliste littéraire qui a écrit environ deux cents chronicles.

Il meurt paralysé et fou dans une maison de santé à Paris en 1893.
\end{document}
