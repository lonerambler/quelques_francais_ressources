\documentclass[a5paper,notitlepage]{article}
\usepackage[utf8]{inputenc}
\usepackage[francais]{babel}
\usepackage[T1]{fontenc}
\usepackage[margin=1in]{geometry}

\author{Brigitte Faucard-Martinez}
\title{Jules Verne}

\begin{document}
\maketitle
Jules Verne naît le 8 février 1828 à Nantes. Vingt ans plus tard, il s'installe à Paris pour commencer ses études de droit et
suivre 1a tradition familiale : son père est en effet un célèbre avocat. Mais Jules Verne n'a qu'une idée en tête : écrire.

Il commence par le théâtre et, grâce à sa rencontre avec Alexandre Dumas, sa comédie \emph{Les Pailles rompues} peut être jouée.

Tout en continuant à travailler pour le théâtre, Jules Verne écrit ses premiers romans. En 1862, il publie \emph{Cinq semaines
en ballon}. Cette \oe{}uvre connaît immédiatement un grand succès.

Encouragé par ces résultats, Jules Verne ne cesse alors de travailler. \emph{Les Aventures du capitaine Hatteras} (1864),
\emph{Les Enfants du capitaine Grant} (1867-1868), \emph{Vingt mille lieues sous les mers} (1870), \emph{Le Tour du monde en
quatre-vingts jours} (1873), \emph{Un capitaine de quinze ans} (1878), \emph{Deux ans de vacances} (1888) et bien d'autres
romans sont publiés pour 1a grande joie de ses lecteurs.

Il meurt à Amiens le 24 mars 1905.

Dans \emph{Vingt mille lieues sous les mers}, Jules Verne nous fait voyager dans le monde mystérieux et fascinant, des océans.
Tout au long du roman, l'auteur nous fait découvrir une faune\footnote{Faune : animaux de la mer.} et une flore\footnote{Flore :
plantes de la mer.} à la fois extraordinaires et fantastiques.

Mais ce roman ne serait pas complet sans son personnage mythique, le capitaine Nemo, inventeur génial du sous-marin \emph{Le
Nautilus}, personnage que Jules Verne nous permet de retrouver dans un autre de ces romans : \emph{L'Île mystérieuse}.

L'image du capitaine Nemo, cet homme à la fois humanitaire et misanthrope\footnote{Misanthrope : personne qui n'aime pas la
compagnie des autres.}, marque en effet chaque page du roman ; c'est à travers lui que les secrets de la mer nous sont dévoilés
et, tout comme Aronnax, le narrateur de l'histoire, nous avons souvent le sentiment, au cours de notre lecture, d'être \og
prisonnier \fg{} de Nemo et de son \emph{Nautilus}.
\end{document}
